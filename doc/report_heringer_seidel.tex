\documentclass{article}

% if you need to pass options to natbib, use, e.g.:
%     \PassOptionsToPackage{numbers, compress}{natbib} before loading
%     neurips_2021

% ready for submission
\usepackage[preprint]{neurips_2021}

% to compile a preprint version, e.g., for submission to arXiv, add add the
% [preprint] option: \usepackage[preprint]{neurips_2021}

% to compile a camera-ready version, add the [final] option, e.g.:
%     \usepackage[final]{neurips_2021}

% to avoid loading the natbib package, add option nonatbib:
%    \usepackage[nonatbib]{neurips_2021}

\usepackage[utf8]{inputenc} % allow utf-8 input
\usepackage[T1]{fontenc}    % use 8-bit T1 fonts
\usepackage[colorlinks=true]{hyperref}       % hyperlinks
\usepackage{url}            % simple URL typesetting
\usepackage{booktabs}       % professional-quality tables
\usepackage{amsfonts}       % blackboard math symbols
\usepackage{nicefrac}       % compact symbols for 1/2, etc.
\usepackage{microtype}      % microtypography
\usepackage{xcolor}         % colors
\usepackage{graphicx}
\setcitestyle{numbers,square}

\title{Analyzing the Enron mails}

% The \author macro works with any number of authors. There are two commands
% used to separate the names and addresses of multiple authors: \And and \AND.
%
% Using \And between authors leaves it to LaTeX to determine where to break the
% lines. Using \AND forces a line break at that point. So, if LaTeX puts 3 of 4
% authors names on the first line, and the last on the second line, try using
% \AND instead of \And before the third author name.

\author{%
  Peter Heringer\\
  Matrikelnummer 6109174 \\
  \texttt{peter.heringer@student.uni-tuebingen.de} \\
  \And Felix Seidel\\
  Matrikelnummer 5969276 \\
  \texttt{felix.seidel@student.uni-tuebingen.de} \\
}

\begin{document}

\maketitle

\begin{abstract}
  The Enron mail corpus provides real world corporate mail communication data.
  The corpus contains mails by Enron personell during the phase in which Enron
  declared bankruptcy. We extract the core working hours of the Enron personell
  from this dataset. The dataset indicates that Enron performed a migration of
  their mail systems shortly before the scandal became public, which introduces
  an interesting shift in the mails' time information. However, we show that
  there is no significant change in core working hours over time. \footnote{The
  git-repository for this project can be found at
  \url{https://github.com/gnampfelix/enron-data-literacy}.} 
\end{abstract}

\section{Introduction}
After the US energy cooperation Enron declared bankruptcy in 2001
\citep{10.1257/089533003765888403}, law enforcement agencies released enormous
amounts of e-mail communication between Enron employees to the public. This raw
database export was cleaned up as the \emph{Enron Mail Corpus} by
\citet{Klimt2004IntroducingTE}. Today, this dataset contains over 500M e-mails
from the mailboxes of Enron employees and can be accessed via
\url{https://www.cs.cmu.edu/~enron/}. This dataset is used in various research
projects, e.g.~for e-mail classification research by
\citet{10.1007/978-3-540-30115-8_22} or graph analysis by
\citet{Chapanond_2005}. The e-mails contained in this dataset span over multiple
years, and interestingly, also contain e-mail conversations throughout 2000 and
2001. As the Enron scandal enrolled during 2001, this period is interesting in
order to check if Enron employees worked differently, i.e.~worked more or less.

\section{Analyzing the time data}
For our work we want to investigate whether there was any significant change in
the amount that employees worked during the time they declared bankruptcy.
This of course is no proof of a direct causal link between these two events, but
makes one at least more plausible.

To analyze this hypothesis, one needs to infer the probable working time from the
times the e-mails were sent. Thus, the first task is to get these timestamps. As
the files contained the headers of the e-mails, the timestamps can be read from
the \texttt{Date} header \citep{rfc5322}. As we only care for the working hours
of Enron employees (and not of those outside people who's mails just happen to
be included in the Enron Mail Corpus), we only consider the e-mails that are
actually sent by \texttt{@enron.com} addresses. Mails from Enron employees that
are sent \emph{to} Enron employees can be included multiple times in the Mail
Corpus. The \texttt{Date} header is usually set by the sender \citep{rfc2821}.
This can be the sender's client or the e-mail server, but never the receiver.
Thus, we consider all e-mails with the same sender, subject and \texttt{Date}
header to be the same.

In Figure \ref{fig:allmail} all mails that are left by removing such duplicates
are drawn. During 1999, the amount of e-mails is small and it increases
drastically by January 2001. During this period, the majority of e-mail
communication occurs between 06:00 and 20:00 and night times are clearly
reflected in the amount of e-mail traffic. Starting February 2001, the silence
in the night disappears and the data is becoming more noisy. From August 2001,
we start to see a separation between working hours and non-working hours again,
but the time frame appears to be shifted +5 hours in comparison to the
pre-February period.

The reason for this shift can be found in the content of the e-mails: Starting
in February 2001, Enron performed a migration from their Lotus Notes e-mail
infrastructure to a Microsoft Outlook infrastructure. Each employee received a
couple of e-mails from \texttt{team.outlook@enron.com}, informing them of their
individual migration date. On migration day, the e-mails of the last 30 days of
the employee were moved to the new infrastructure automatically. Each employee
was responsible to manually migrate older e-mails that they need. When looking
at the mailbox of individual employees, one can find duplicates that seem to be
created during this period. Interestingly, the duplicates are the exact same
e-mails, but have a \texttt{Date} header that is shifted by exactly five, six,
ten or twelve hours. We can only speculate what the precise reason for this
shift is. One possible explanation is a misconfiguration of the different e-mail
databases from which the dataset was exported.

This is why the dataset as parsed for Figure \ref{fig:allmail} is not suitable
to infer the core working hours directly, especially in the migration period:
Duplicate mails with shifted time information that might be even duplicated and
shifted multiple times more, due to them being present in multiple mailboxes
that were migrated on different days, make the results virtually unusable.

\begin{figure}
  \centering
  \includegraphics[width=1.0\linewidth]{../src/fig001_mail_all.pdf}
   \caption{All mails in the Enron Mail Corpus that are sent by
  \texttt{@enron}.com addresses. The vertical red dashed line marks the beginning of
  the migration period with the usual start of the core working hours indicated
  by the horizontal red dashed line. The vertical blue dashed line marks the
  assumed end of the migration period with the new usual start of the core
  working hours indicated by the horizontal blue dashed line.}
  \label{fig:allmail}
  \vspace{-4mm}
\end{figure}

To reduce this noise, we limit our e-mail parsing to the folders
\texttt{sent\_items}, \texttt{\_sent\_mail}, \texttt{\_sent} and \texttt{sent}.
By doing this, we naturally remove many duplicates, as only one mailbox can be
the sender mailbox of a particular e-mail. Furthermore, we exclude sender
addresses that are not represented with a mailbox in the dataset and would
therefore only provide few data points. The remaining duplicates are removed by
considering e-mails with the same sender, subject, minute and second as
duplicate. This respects the time being shifted in some of the duplicates by
exactly five, six, ten or twelve hours.

The result of this is visualized in Figure \ref{fig:jeffmail}. By limiting the
parsing to those four folders and by removing some more duplicates, we are able
to reduce the noise drastically.

\begin{figure}
  \centering
  \includegraphics[width=1.0\linewidth]{../src/fig002_mail_jeff.pdf}
  \label{fig:jeffmail}
  \caption{All mails in the Enron Mail Corpus that are sent by
  \texttt{jeff.dasovich@enron.com}. In all four plots, the red dashed line marks
  the begin of Jeff's e-mail migration period, the blue dashed line the end. The
  black dashed line marks the end of the 2001 daylight saving time in the USA.
  The upper row shows all e-mails in the mail corpus where the sender is
  \texttt{jeff.dasovich@enron.com}, the lower row only those e-mails by
  \texttt{jeff.dasovich@enron.com} that are in a \texttt{.*sent.*}-folder. The
  left column considers e-mails with the same subject, sender, time and date
  duplicate. The right column considers e-mails with the same subject, sender,
  minute and second duplicate.}
\end{figure}

When using this de-duplicated dataset and plotting boxplots (Figure
\ref{fig:mailboxplot}) for each employee, one still can see big differences
between the individual employees. This might have different reasons like
employees having different core working hours, flexible time management, working
only before or after the e-mail migration or working in different timezones. The
boxplot uses the 25th and 75th quantile for the box limits, which is reasonable
to us. We will use those quantiles to calculate the core working hours for each
employee in each month. For each month, we will exclude employees who wrote less
than 10 e-mails in that month, as for those, the working hours cannot be
inferred. 

\begin{figure}
  \centering
  \includegraphics[width=1.0\linewidth]{../src/fig003_mails_boxplott.pdf}
   \caption{The mail communication for 20 arbitrary Enron employees, the box
   limits are the 25th and the 75th quantile.}
  \label{fig:mailboxplot}
\end{figure}

We test whether we can reject $H_0$, that the amount of hours worked by
the Enron staff has not changed significantly over time. We assume the first
half of 2000 to be our baseline, as there are only few data points before that
date in the Enron Mail Corpus. We re-sampled that baseline 10000 times for each
following month with a sample size equal to the amount of data points for that
month and checked whether the average amount of hours worked is more extreme on
both sides than the average amount of hours worked inferred from the data of
that month. An excerpt of this data is depicted in Table \ref{ta:pvalues}. We
found no significant divergence from the baseline for any month in the dataset.

\begin{table}
  \caption{The average working hours per month. The first half of 2000 acts as a
  baseline against which the following months are checked for significant
  divergence.}
  \centering
  \begin{tabular}{lcccccccc}
    \toprule
    \textbf{month} & H1 00 & 06/01 & 07/01 & 08/01 & 09/01 & 10/01 & 11/01 &
    12/01 \\
    \midrule
    \textbf{avg. hours} & 05:00 & 05:45 & 05:16 & 05:43 & 04:48 & 05:00 & 05:12
    & 04:30 \\
    \textbf{p-Value} & - & 0.524 & 0.530 & 0.522 & 0.503 & 0.525 & 0.521 & 0.520
    \\
    \bottomrule
  \end{tabular}
  \label{ta:pvalues}
  \vspace{-4mm}
\end{table}

\section{Discussion}
We've made some assumptions for our analysis that we want to address briefly.
Firstly, we assumed that we can infer the core working hours from the mail data.
Every individual person might start or end their working day differently, e.g.~
by participating in daily meetings (no e-mail data at all) or by answering to
e-mails of the last day (very much e-mail data). In general, we assumed that
each working day has a well-defined start and end and did not consider more
flexible arrangements (e.g.~to watch over the kids during an extended lunch
break). Also, there might be differences in the working hours that occur
periodically (e.g.~such as working less on Friday).

Secondly, the amount of e-mails sent by individuals depends in parts on the
amount of e-mails they receive, i.e.~an answer $B$ to an e-mail $A$ can naturally
only be written after the initial e-mail $A$ was received. We acknowledge this
by looking at the average working time for all employees in each month in terms
of the 25th and the 75th quantile.

Finally, we base our analysis on the assumption, that the time and date
reconstruction from the original database export was performed consistently
according to the appropriate RFC 5322 \citep{rfc5322} and RFC 2821
\citep{rfc2821}, which might not have been the case.

\bibliographystyle{plainnat}
\bibliography{lib}

\end{document}
